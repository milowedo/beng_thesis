\chapter{Wprowadzenie}
\label{cha:wprowadzenie}

Poniższa praca prezentuje projekt i wykonanie systemu składającego się z kilku osobno zaimplementowanych serwisów połączonych w aplikacji mobilnej.
Stawia on sobie na cel ułatwienie oraz usprawnienie kompletowania domowej bilbioteki poprzez analizę ofert sprzedawców na platformie aukcyjnej Allegro.pl.\newline
\newline
%---------------------------------------------------------------------------

\section{Temat pracy}
\label{sec:tematPracy}

Tematem pracy jest aplikacja mobilna napisana w frameworku React Native, która deleguje potrzebne funkcjonalności do zewnętrznych serwisów. Jej architekturę określić można jako rozproszoną, stąd możliwym jest rozbudowywanie poszczególnych usług niezależnie od innych. Dzięki takiemu podejściu podczas rozwijania systemu, jego twórca nie musi panicznie martwić się o zasoby platformy i ograniczać na tej podstawie tworzonych funkcjonalności. Dodatkowo, pojedyncze elementy struktury mogą być zaimplementowane w dowolnym języku.\newline
Sam system zajmuje się analizą dostępnych ofert książek na stronie Allegro.pl. Ma to na celu optymalizację zakupów, których zamierzeniem jest wejście w posiadanie jak największej ilości poszukiwanych książek po możliwie najniższym koszcie.\newline
Jako wynik dostarczana jest kolekcja o charakterze klucz-wartość, gdzie kluczem jest sprzedawca na platformie aukcyjnej Allegro.pl, a wartością zbiór książek, które posiada on w swojej ofercie, a które znajdują się jednocześnie na liście szukanych przez użytkownika pozycji.
%---------------------------------------------------------------------------
\newpage
\section{Motywacja}
\label{sec:motywacja}
Pomysł na stworzenie tego typu aplikacji powstał podczas przeszukiwania serwisu Allegro.pl w celu znalezienia interesujących wtedy, dla autora tej pracy, książek. Problem jaki został napotkany polegał na tym, że w momencie skompletowania zestawu artykułów, okazało się, że ceny wysyłek znacząco podwyższają finalną cenę zamówienia. Najlepszym rozwiązaniem zdawało się znalezienie ofert jednego sprzedawcy, dzięki czemu za transport zapłaconoby raz. Niestety, na wspomnianej platformie aukcyjnej, użytkowników mających w swojej ofercie książki jest sporo. Analizowanie wszystkich przedmiotów u wszystkich ich posiadaczy wymaga poważnej ilości czasu, którego poświęcenie mogłoby ostatecznie okazać się niezbyt opłacalne.
Niestety, dostępne na rynku aplikacje nie realizują w sposób satysfakcjonujący funkcjonalności, które rozwiązywałyby napotkany problem.

%---------------------------------------------------------------------------

\section{Cele pracy}
\label{sec:celePracy}
\begin{enumerate}
    \item Przygotowanie schematu architektury systemu
    \item Implementacja poszczególnych serwisów
    \item Dostarczenie aplikacji, która: 
    \begin{itemize}
        \item Daje możliwość zarejestrowania, zalogowania oraz wylogowania
        \item Ma zabezpieczone zasoby
        \item Jest chroniona przed nieautoryzowanym dostępem
        \item Posiada osobny serwis służący do autoryzacji i autentykacji
        \item Umożliwia zapisywanie list książek powiązanych z użytkownikiem w zewnętrznej bazie danych 
        \item Asynchronicznie pobiera dane i przelicza oferty prezentowane użytkownikowi
        \item Wizualizacje danych przedstawia w postaci przyjaznej dla odbiorcy
        \item Jej część mobilną charakteryzuje płynność, przez co użytkownik nie doświadcza nieprzyjemności związanych z jej obsługą
    \end{itemize}
\end{enumerate}
%---------------------------------------------------------------------------
\newpage
\section{Zawartość pracy}
\label{sec:zawartoscPracy}
W rozdziale \textit{Wprowadzenie} omówiono temat pracy oraz motywacje jakie stoją za implementacją tego konkretnego rozwiązania. Wspomniano o braku gotowych aplikacji realizujących zadane funkcjonalności oraz wylistowano cele jakie stawia sobie poniższa praca.\newline
W rozdziale 2 omówiono szczegółowo sprawy związane z architekturą systemu. Przedstawiono podejście jakim kierowano się w procesie rozwijania produktu. Załączony został schemat, w celu wizualizacji struktury systemu. W kolejnych podrozdziałach opisano funkcje poszczególnych serwisów, starając się nakreślić cechy niektórych ich aspektów oraz wytłumaczyć ważniejsze pojecia. Przeanalizowane zostały różne podejścia do tworzenia oprogramowania a także wartości, które pozytywnie mogłyby wpłynąć na końcowy odbiór produktu.\newline
Na początku rodziału \textit{Implementacja} zawarty jest podrozdział traktujący o wykorzystanej metodyce pracy, która umożliwiła przejrzyste zorganizowane zadań i kontrolę postępów. Następnie opisane zostały technologie, użyte w implementacji usług realizujących zadane funkcjonalności. Omówiono również bliżej niektóre rozwiązania, które autor pracy uznał za szczególnie warte wspomnienia. W ostatnim podrozdziale znalazły się informacje na temat wdrożenia poszczególnych elementów systemu.\newline
W rozdziale 4 opisane zostały poszczególne ekrany aplikacji mobilnej z uwzględnieniem funkcjonalności dostępnych z punktu widzenia użytkownika.\newline
W ostatnim, 5 rozdziale zawarte jest podsumowanie wykonanej pracy oraz zaprezentowane są możliwe udoskonalenia systemu.
%-------------------------------------