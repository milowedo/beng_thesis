\chapter{Wprowadzenie}
\label{cha:wprowadzenie}

Poniższa praca prezentuje projekt i wykonanie systemu składającego się z kilku osobno zaimplementowanych serwisów połączonych w aplikacji mobilnej.\\
Stawia on sobie na cel ułatwienie oraz usprawnienie kompletowania domowej bilbioteki.
%---------------------------------------------------------------------------

\section{Temat pracy}
\label{sec:tematPracy}

Tematem pracy jest aplikacja mobilna napisana w frameworku React Native, która deleguje potrzebne funkcjonalności do zewnętrznych serwisów. Jej architekturę określić można jako rozproszoną, stąd możliwym jest rozwijanie poszczególnych usług niezależnie od innych. Dzięki takiemu podejściu nie jest najważniejszym troska o zasoby platformy, a pojedyncze elementy struktury mogą być zaimplementowane w dowolnym języku.\\
Sam system zajmuje się analizą dostępnych ofert książek na stronie Allegro.pl. Ma to na celu optymalizację zakupów użytkownika, którego zamierzeniem jest wejście w posiadanie jak największej ilości poszukiwanych książek po możliwie najniższym koszcie.


%---------------------------------------------------------------------------

\section{Motywacja}
\label{sec:motywacja}
Pomysł na stworzenie tego typu aplikacji powstał podczas przeszukiwania serwisu Allegro.pl w celu znalezienia interesujących wtedy, dla autora tej pracy, książek. Problem jaki został napotkany polegał na tym, że w momencie skompletowania zestawu artykułów, okazało się, że ceny wysyłek znacząco podwyższają finalną cenę zamówienia. Najlepszym rozwiązaniem zdawało się znalezienie ofert jednego sprzedawcy, dzięki czemu za transport zapłaconoby raz. Niestety, na wspomnianej platformie aukcyjnej, użytkowników mających w swojej ofercie książki, jest dużo. Analizowanie wszystkich przedmiotów u wszystkich ich posiadaczy wymaga poważnej ilości czasu, którego poświęcenie mogłoby ostatecznie okazać się niezbyt opłacalne.\\
Dostępne na rynku aplikacje nie realizują w sposób satysfakcjonujący funkcjonalności, które rozwiązywałyby napotkany problem.

%---------------------------------------------------------------------------

\section{Cele pracy}
\label{sec:celePracy}
\begin{enumerate}
    \item Przygotowanie schematu architektury systemu
    \item Implementacja poszczególnych serwisów
    \item Dostarczenie aplikacji, która: 
    \begin{itemize}
        \item Ma zabzpieczone zasoby i ochronę przed nieautoryzowanym dostęptem.
        \item Umożliwia zapisywanie list książek w zewnętrznej bazie danych 
        \item Asynchronicznie pobiera dane i przelicza oferty prezentowane użytkownikoiw
        \item Wizualizacje danych przedstawia w postaci przyjaznej dla użytkownika
        \item Część mobilną posiada płynną i dobrze zoptymalizowaną, przez co użytkownik nie ma doświadcza nieprzyjemności związanych z jej obsługą.
    \end{itemize}
\end{enumerate}

%---------------------------------------------------------------------------

\section{Zawartość pracy}
\label{sec:zawartoscPracy}
W rozdziale \textit{Wprowadzenie} omówiono temat pracy oraz motywacje jakie stoją za implementacją tego konkretnego rozwiązania. Wspomniano o braku gotowych aplikacji realizujących zadane funkcjonalności oraz wylistowano cele jakie stawia sobie poniższa praca.
\\W rozdziale 2 omówiono szczegółowo sprawy związane z architekturą systemu. Przedstawiono podejście jakim kierowano się w procesie rozwijania produktu. Załączony został schemat struktury, w celu wizualizacji struktury systemu. W kolejnych podrozdziałach opisano funkcje poszczególnych serwisów, starając się wytłumaczyć ważniejsze pojecią i nakreślić cechy niektórych ich aspektów. Przeanalizowane zostały różne podejścia do tworzenia oprogramowania\\ a także wartości, które pozytywnie mogłyby wpłynąć na końcowy odbiór produktu.
\\W rozdziale \textit{Implementacja} zawarty jest podrozdział traktujący o wykorzystanej metodyce pracy, która umożliwiła przejrzyste zorganizowane zadań i kontrolę postępów. Następnie opisane zostały technologie użyte w implementacji usług realizujących zadane funkcjonalności. Wspomniano bliżej niektóre elementy, które autor pracy uznał za bardziej ciekawe. W ostatnim podrozdziale znalazły się informacje na temat wdrożenia poszczególnych elementów systemu.
\\W rozdziale 4 opisane zostały poszczególne ekrany aplikacji mobilnej z opisaniem funkcjonalności dostępnych z punktu widzenia użytkownika.
\\W ostatnim, 5 rozdziale zawarte jest podsumowanie wykonanej pracy oraz zaprezentowane są możliwości rozwoju.
%-------------------------------------