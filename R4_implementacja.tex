\chapter{Implementacja}
\label{cha:implementacja}

tutaj gadane jakiś ogólny opis i w ogóle cnie

%---------------------------------------------------------------------------

\section{Wybór technologii}

%---------------------------------------------------------------------------

\section{Wielowątkowe tworzenie ofert}

%---------------------------------------------------------------------------

\section{Autoryzacja użytkownika w Allegro}

Do integracji serwisu z aplikacją potrzebne jest pozyskanie tokenu dostępowego. Allegro udostępnia tzw. ``ścieżkę device flow``, dzięki której cały proces odbywa się bez konieczności uwzględniania go w interfejsie. Podejście tutaj zaprezentowane bazuje na zarejestrowaniu jednego, wspólnego dla całej aplikacji, konta funkcjonalnego za pomocą którego każde zapytanie będzie się autentykować. Uzyskane zostaną dwa tokeny: 
\begin{itemize}
	\item dostępowy - ważny przez 12h.
	\item odświeżający - ważny 6 miesięcy.
\end{itemize}
W momencie 

%---------------------------------------------------------------------------


\section{Elementy konifguracyjne}

%---------------------------------------------------------------------------

\section{Wdrożenie}

%---------------------------------------------------------------------------