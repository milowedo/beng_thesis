\chapter{Podsumowanie}
\label{cha:podsumowanie}

Projekt i implementacja przedstawionego w tej pracy systemu zostały wykonane w sposób wystarczająco zadowalający. Użytkownik może w zaledwie parę sekund otrzymać wyniki przeanalizowania setki możliwych ofert od sprzedawców na platformie Allegro.pl. Aplikacja jest w stanie przetrzymywać listy książek dla użytkowników, w zewnętrznym, zabezpieczonym przed nieautoryzowanym dostępem, archiwum chmurowym. Brak więc obaw o utratę danych z urządzenia mobilnego. Czasochłonne obliczenia i analizy dostępnych w serwisie aukcyjnym przedmiotów wyekstraktowano do osobnego serwisu o zdecydowanie większej mocy obliczeniowej niż przeciętny komputer podręczny.
%---------------------------------------------------------------------------

\section{Wnioski}
W procesie tworzenia aplikacji dzięki podjętym decyzjom i rozwiązanym problemom, autorowi tej pracy udało się nabyć cenne doświadczenia. 
Użycie nierelacyjnej bazy danych można określić jako rozwiązanie trafne i wydajne. Również integracja z zewnętrznym API platformy Allegro.pl jest najtrafniejszą decyzją, jaką autor mógł podjąć, decydując się na źródło danych dla aplikacji.
Sporym wyzwaniem było na pewno połączenie poszczególnych serwisów wdrożonych jako osobne aplikacje w jeden, komunikujący się między sobą twór.
System jest zabezpieczony przed nieautoryzowanym dostępem. Jego budowa to luźno powiązane elementy, które można zamieniać i modyfikować bez destrukcyjnego wpływu na działanie całości aplikacji.\\
W łatwy sposób można go rozszerzyć, dodając kolejne serwisy i włączając je w odpowiednich miejscach.

\newpage
\section{Możliwe rozszerzenia i usprawnienia}
Zdecydowanie ciekawym usprawnieniem byłaby na przykład możliwość łączenia baz poszukiwanych książek z bilbiotekami innych użytkowników, czy większa personalizacja wyszukiwań pod kątem chociażby czasowego wykluczania niektórych pozycji z analizy.\\
Bazując również na posiadanych tomach, warte byłoby rozważenie algorytmów, które potrafiłyby wskazać użytkownikowi rekomendowane przez system pozycje.\\
Ze względu na to, że struktura systemu złożona z osobnych serwisów pozwala w łatwy sposób zaadaptować dowoloną ilość nowych funkcjonalności, stworzona aplikacja ma spory potencjał na rozwój.