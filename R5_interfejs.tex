\chapter{Interfejs}
\label{cha:interfejs}

Interfejs aplikacji składa się z trzech rodzielnych zbiorów ekranów. Każda nawigacja pomiędzy tymi grupami skutkuje usunięciem z pamięci kolekcji ekranów, znajdujących się \\w poprzedniej grupie oraz załadowaniem nowego zestawu.\\
W pierwszym pakiecie znajdują się ekrany logowania i rejestracji. Jeżeli użytkownik się zaloguje lub zarejestruje, jego token dostępowy zostanie zapisany w pamięci podręcznej urządzenia i do momentu jej wyczyszczenia, program nie będzie wymagał od niego ponownego wpisywania swoich danych w celu autoryzacji. Co za tym idzie, w ogóle nie zaistnieje potrzeba załadowania tych widoków.
Po autoryzacji użytkownik otrzymuje dostęp do drugiego i trzeciego zbioru zawierających :
\begin{enumerate}
    \setcounter{enumi}{1}
    \item biblioteki książek oraz widok z zaprezentowanymi ofertami
    \item ustawienia
\end{enumerate} 

\begin{figure}[H]
	\centering
	\includegraphics[width=\linewidth]{navig.pdf}
	\caption{Schemat nawigacji pomiędzy ekranami}
\end{figure}
%---------------------------------------------------------------------------

\section{Logowanie i rejestracja}
Te dwa ekrany zawierają formularze w których użytkownik może wpisać email oraz hasło. Po wpisaniu danych, akceptuje formularz niebieskim przyciskiem i tworzy zapytanie do Auth Service. W sytuacji, gdy wprowadzone dane są niewłaściwe, zostanie wyświetlony odpowiedni komunikat.\\
Na dole ekranu widnieje krótki tekst, po którego kliknięciu, nastąpi zresetowanie formularza\\ i przeniesienie do sąsiedniego widoku.

\begin{figure}[H]
	\centering
	\includegraphics[width=\linewidth]{signin_signup.pdf}
	\caption{Ekrany logowania i rejestracji w aplikacji mobilnej}
\end{figure}
%---------------------------------------------------------------------------
\section{Ekrany bibliotek}
Widoki \textit{Wanted} i \textit{My library} korzystają w większości z tych samych komponentów, różni je natomiast sposób oraz cel ich użycia. Obydwa zawierają listy, których jedynie widoczne elementy są renderowane. Można je przesuwać werykalnie, a każdy element posiada ukryte opcje z każdej strony, które można aktywować poprzez horyzontalne przesunięcie.

\newpage
Widok poszukiwanych książek prezentuje pozycje, które będą wysłane do serwisu OffersFetcher i użyte w celu stworzenia ofert.\\
Na poniższej grafice widnieje funkcjonalność dodawania nowej pozycji do listy. Po naciśnięciu  obszaru z napisem ``new book``, wysunięty zostanie formularz, który poprawnie wypełniony poskutkuje dodaniem nowej książki i wysłaniem jej do bazy danych w chmurze. Cenę każdej pozycji można edytować po jej naciśnięciu - pojawi się pole do edytowania wartości.
\begin{figure}[H]
	\centering
	\includegraphics[width=\linewidth]{wanted.pdf}
	\caption{Biblioteka poszukiwanych książek oraz dodawanie nowej pozycji}
\end{figure}
\newpage

Poprzez przesunięcie pojedynczego elementu w lewo, pojawi się ukryty pod spodem przycisk, który służy do usunięcia danej książki z listy i bazy danych.\\ Jeżeli bloczek poruszony zostanie ruchem o przeciwnym zwrocie, użytkownik otrzyma możliwość edytowania informacji o danym tomie.\\
Każde wychylenie elementu zostanie przywrócone do stanu wyjściowego w momencie poruszenia innego lub po 5 sekundach bezczynności.
\begin{figure}[H]
	\centering
	\includegraphics{mylib.pdf}
	\caption{Bilbioteka posiadanych książek oraz funkcjonalność usuwania}
\end{figure}
\newpage
%---------------------------------------------------------------------------

\section{Ekran z ofertami}
To tutaj zaprezentowane są wyniki analiz wykonanych w serwisie OffersFetcher.\\ Jest to przesuwalny wertykalnie komponent zawierający sprzedawców oraz ich produkty, na bazie pozycji z ekranu Wanted. Każdy element składa się z identyfikatora właściciela aukcji, następnie z listy książek, gdzie każdy obiekt to zdjęcie prezentujące produkt, tytuł, autor, a także jego cenę. Na dole oferty wyświetlona jest sumaryczna cena tomów oraz najtańsza możliwa dostawa według kontrahenta.
\begin{figure}[H]
	\centering
	\includegraphics[width=\linewidth]{offers.pdf}
	\caption{Ekran prezentujący oferty od sprzedawców}
\end{figure}
\newpage

\section{Opcje}
Obecnie dostępna jest tylko jedna opcja - mianowicie wylogowanie użytkownika. Po naciśnięciu przycisku, usunięty zostanie token dostępowy, a aplikacja wykona przeniesienie\\ do ekranu logowania.
\begin{figure}[H]
	\centering
	\includegraphics{options.pdf}
	\caption{Ekran opcji z możliwością wylogowania}
\end{figure}

